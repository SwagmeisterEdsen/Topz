A procedure is an executable chunk of code.

\begin{table}[h]
    \centering
    \label{tbl:procedure}
    \begin{tabular}{|l|l|l|}
        \hline
        \textbf{Offset} & \textbf{Size in Bytes} & \textbf{Description}                                                    \\ \hline
        0               & 1                      & Boolean telling if this procedure is the starting point of the program. \\ \hline
        1               & 4                      & The size of the code in the procedure.                                  \\ \hline
        5               & x                      & The executable code of the procedure.                                   \\ \hline
        x               & x                      & Number of references to other atoms.                                    \\ \hline
        x               & x                      & The references to other atoms.                                          \\ \hline
    \end{tabular}
\end{table}

\subsection{Is Main}
If true the procedure is the starting point of the program. It is not required 
that an object file has a main procedure, but there cannot be more than one.

\subsection{Size of the Procedure}
This is the size of the code block following this number, in bytes. This is an
usinged 32 bit integer.

\subsection{Code}
The actual code of the procedure.

\subsection{Number of References}
An unsigned 16 bit number telling how many references this procedure has to 
other atoms.

\subsection{References}
The references defines where the procedure refer to other atoms, the size of the 
address and if it the address is in little endian.

\begin{table}[h]
    \centering
    \label{tbl:reference}
    \begin{tabular}{|l|l|l|}
        \hline
        \textbf{Offset} & \textbf{Size in Bytes} & \textbf{Description}                             \\ \hline
        0               & 4                      & The number of the atom that is being referenced. \\ \hline
        4               & 1                      & Boolean telling if the address is little endian  \\ \hline
        5               & 1                      & Address size in bytes.                           \\ \hline
        6               & 4                      & The address to relocate.                         \\ \hline
    \end{tabular}
\end{table}

\subsubsection{Number of the Referenced Atom}
The number of an atom is the number of atoms declared before the referenced one.
This number is used instead of the name to save space and improve performance. 
This number is an unsigned 32 bit integer.

\subsubsection{Is it in Little Endian}
If true the address is in little endian.

\subsubsection{Address Size}
The size of an address can be 2, 4 or 8 bytes long.

\subsubsection{Address}
This is the address within the code block of the procedure to relocate to match
the address of the referenced atom.