The object file consist of \textbf{atoms}. An atom is an indivisible chunk of
code or data. There are a few types of atoms, but they all have a set of common
fields. These fields are:

\begin{table}[h]
    \centering
    \label{tbl:atom}
    \begin{tabular}{|l|l|l|}
        \hline
        \textbf{Offset} & \textbf{Size in Bytes} & \textbf{Description}                                            \\ \hline
        0               & 1                      & The type of atom.                                               \\ \hline
        1               & 1                      & Boolean telling if the atom is defined.                         \\ \hline
        2               & 1                      & Boolean telling if the atom is available to other object files. \\ \hline
        14              & x                      & The name of the atom in utf8.                                   \\ \hline
    \end{tabular}
\end{table}

All types of atoms appends their data to the data mentioned in the table above.

\subsection{Types}
This is an unsinged byte telling the which type of atom it is.

\begin{table}[h]
    \centering
    \label{tbl:type}
    \begin{tabular}{|l|l|}
        \hline
        \textbf{Number} & \textbf{Type}          \\ \hline
        0               & Procedure              \\ \hline
        1               & Null terminated string \\ \hline
        2               & Data                   \\ \hline
    \end{tabular}
\end{table}

\subsection{Is Defined}
If true this atom is some code or data otherwise; a reference to an atom in 
another object file.

\subsection{Is Global Atom}
If true outside atoms is allowed to reference this atom; otherwise only atoms in
this atom's object file may reference it.

\subsection{Names}
The atom's name is a null terminated string encoded in utf8. The name is case 
sensitive and it must be unique.